%%%%%%%%%%%%%%%%%%%%%%%%%%%%%%%%%%%%%%%%%%%%%%%%%%%%%%%%%%%%%%%%%%%%%%
% LaTeX Template: Newsletter  % Source: http://www.howtotex.com
%
% Feel free to distribute this example, but please keep the referral
% to howtotex.com
% Date: September 2011 
% 
%%%%%%%%%%%%%%%%%%%%%%%%%%%%%%%%%%%%%%%%%%%%%%%%%%%%%%%%%%%%%%%%%%%%%%
% How to use writeLaTeX: 
%
% You edit the source code here on the left, and the preview on the
% right shows you the result within a few seconds.
%
% Bookmark this page and share the URL with your co-authors. They can
% edit at the same time!
%
% You can upload figures, bibliographies, custom classes and
% styles using the files menu.
%
% If you're new to LaTeX, the wikibook is a great place to start:
% http://en.wikibooks.org/wiki/LaTeX
%
%%%%%%%%%%%%%%%%%%%%%%%%%%%%%%%%%%%%%%%%%%%%%%%%%%%%%%%%%%%%%%%%%%%%%%
% Edit the title below to update the display in My Documents
%\title{Newsletter Template}

%%% ---------------
%%% PREAMBLE
%%% ---------------
\documentclass[10pt,a4paper]{article}

% Define geometry (without using the geometry package)
\setlength\topmargin{-48pt}
\setlength\headheight{0pt}
\setlength\headsep{25pt}
\setlength\marginparwidth{-20pt}
\setlength\textwidth{7.0in}
\setlength\textheight{9.5in}
\setlength\oddsidemargin{-30pt}
\setlength\evensidemargin{-30pt}

\frenchspacing						% better looking spacing

% Call packages we'll need
\usepackage[frenchb]{babel}			% english
\usepackage{graphicx}				% images
\usepackage{amssymb,amsmath}		% math
\usepackage{multicol}				% three-column layout
\usepackage{url}					% clickable links
\usepackage{marvosym}				% symbols
\usepackage{wrapfig}				% wrapping text around figures
\usepackage[utf8]{inputenc}			% font encoding

\usepackage{tgbonum}



\usepackage{charter} 				% Charter font for main content
\usepackage{blindtext}				% dummy text
\usepackage{datetime}				% custom date
	\newdateformat{mydate}{\monthname[\THEMONTH] \THEYEAR}
\usepackage[pdfpagemode=FullScreen,
			colorlinks=false]{hyperref}	% links and pdf behaviour

% Customize (header and) footer
\usepackage{fancyhdr}
\pagestyle{fancy}
\lfoot{	\footnotesize 
		Science \& Technology Newletter \\
		\Mundus\ \href{http://www.howtotex.com}{HowToTeX.com}	\quad
		\Telefon\ 555-5555											\quad
		\Letter\ \href{mailto:frits@howtotex.com}{frits@howtotex.com}
	  }
\cfoot{}
\rfoot{\footnotesize ~\\ Page \thepage}
\renewcommand{\headrulewidth}{0.0pt}	% no bar on top of page
\renewcommand{\footrulewidth}{0.4pt}	% bar on bottom of page

%%% ---------------
%%% DEFINITIONS
%%% ---------------

% Define separators
\newcommand{\HorRule}[1]{\noindent\rule{\linewidth}{#1}} % Creating a horizontal rule
\newcommand{\SepRule}{\noindent							 % Creating a separator
						\begin{center}
							\rule{250pt}{1pt}
						\end{center}
						}						

% Define Title en News input
\newcommand{\JournalName}[1]{%
		\begin{center}	
			\Huge 
			#1%
		\end{center}	
		\par \normalsize \normalfont}
		
\newcommand{\JournalIssue}[1]{%
		\hfill \textsc{\mydate \today}
		\par \normalsize \normalfont}

\newcommand{\NewsItem}[1]{%
			
		\large \bfseries #1 \vspace{4pt}
		\par \normalsize \normalfont}
		
\newcommand{\NewsAuthor}[1]{%
			\hfill by \textsc{#1} \vspace{4pt}
			\par \normalfont}		


%%% ---------------
%%% BEGIN DOCUMENT
%%% ---------------
\begin{document}
% Title	
% -----

\JournalName{Centre de Mod\'elisation et Simulation de Strasbourg}
\noindent\HorRule{3pt} \\[-0.75\baselineskip]
\HorRule{1pt}
% -----

% Front article
% -----
\vspace{0.5cm}
	\SepRule
\vspace{0.5cm}
\begin{figure}[h]

\begin{center}


		\includegraphics[width=0.42\textwidth]{images/cemosis}
		\\	% this spacer is needed to make the text on the right fit OK


\end{center}
\end{figure}
% -----


% Other news (1)
% -----
\vspace{0.5cm}
	\SepRule
\vspace{0.5cm}
\begin{multicols}{3}
	\NewsItem{Histoire de la structure}
 Cemosis a \'et\'e crée en Janvier 2013, en réponse à l'appel IDEX 2012 pour un projet d'attractivité, dans le but de supporter C. Prud'homme, professeur dans l'équipe modélisation et Contrôle de l'IRMA (Institut de recherche en mathématiques avancées). Le but initial était double : 
 \begin{itemize}
 	\item[-] L'ouverture de nouveaux axes de recherche pluridisciplinaire, en plus de ceux existant déja : fusion et physique des plasmas.
 	\item[-] Le développement des relations entre mathématiques et entreprises.
 \end{itemize}
 
 En 2015, l'étude sur l'impact socio-économique des mathématiques (rapport EISEM), donne un premier avis sur la contribution forte des mathématiques dans l'économie nationale et considère les mathématiques comme jouant un rôle qui adresse les challenges majeurs de demain. Cemosis est cité dans le rapport et regardé comme un modèle d'initiative développant les interactions entre l'industrie et les mathématiques.
 

\vspace{1cm}
% Other news (2)
% -----
\NewsItem{}
\NewsAuthor{J}
%	\blindtext[1]
%		\begin{center}
%			\includegraphics[width=0.8\linewidth]{}
%		\end{center}
%		\blindtext[1]
\end{multicols}
% -----
\end{document} 
